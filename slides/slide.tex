\documentclass{beamer}

\usetheme[progressbar=head, block=fill]{metropolis}
\usepackage[utf8]{inputenc}
\usepackage[makeroom]{cancel}
\usepackage{amsmath}

\usepackage{amsmath}
\usepackage{amsfonts}
\usepackage{mathtools}

\usepackage{siunitx} 

\usepackage{manfnt}
\usepackage{graphicx}
%% \usepackage[dvipsnames]{xcolor}
%% \definecolor{dgreen}{rgb}{0,0.6,0}
\usepackage{amssymb}
\usepackage{pifont}
\usepackage{mathtools}
\usepackage{multimedia}
\usepackage{media9}
\usepackage{centernot}
\usepackage{amsthm}

\graphicspath{ {figures/} }

\renewcommand{\vec}[1]{\boldsymbol{#1}}
\newcommand{\matr}[1]{\mathbf{#1}}
\newcommand{\cmark}{\ding{51}}%
\newcommand{\xmark}{\ding{55}}%

\newcommand{\unitvec}[1]{\hat{\boldsymbol{#1}}}
\newcommand{\transpose}{^{\mathrm{T}}}

\newcommand{\cubicFrame}[1]{{\color{orange}#1}}
\newcommand{\qx}[1]{{\color{red}#1}}
\newcommand{\qy}[1]{{\color{green}#1}}
\newcommand{\qz}[1]{{\color{blue}#1}}

\date{17/05/2017}

\title[]{Controllability, observability and Noninteracting control of the Cubli}
\subtitle{Corso di LM in Ingegneria Robotica ed Automazione \\
  Controllo dei Robot}
\author{Students:\hfill Supervisors:\\
Nicola Piga \hfill Prof. Antonio Bicchi\\
Giulio Romualdi \hfill Ing. Manuel Bonilla}
\institute[]{Università di Pisa}
\begin{document}
%\beamertemplatenavigationsymbolsempty

\begin{frame}
  \maketitle
  %% \hskip-in
  %% \centering
  %% \begin{columns}
  %%   \begin{column}{0.4\columnwidth}
  %%   \end{column}
  %%   \begin{column}{0.265\columnwidth}
  %%     \centering
  %%     \includegraphics[width=20mm]{cherubino}
  %%   \end{column}
  %%   \begin{column}{0.6\columnwidth}
  %%     \vskip.1in
  %%     \hskip.3in
  %%     Supervisors:\\
  %%     \hskip.3in
  %%     Prof. Antonio Bicchi\\
  %%     \hskip.3in
  %%     Ing. Manuel Bonilla\\
  %%   \end{column}
  %% \end{columns}
\end{frame}

\section*{Introduction}
In this report the theory of nonlinear systems is applied to a mechanical system known in the literature
as the Cubli \cite{Gajamohan2013}. It is a reaction wheel based 3D inverted pendulum consisting of a $\SI{15}{\centi\meter}$
sided cube with flywheels mounted on the three faces.
\begin{figure}[h]
  \centering
  \includegraphics[scale=0.5]{cubli.png}
  \caption{The Cubli}
\end{figure}
\par
In the section one the equations of motion of the system are derived in standard manipulator form and are rewritten
in control affine form which is required to apply the theory of nonlinear systems. Also the equilibria of the system
are described.
\par
In the section two and three the theory regarding nonlinear controllability and observability is briefly recalled
and then is applied to the system.
\par
In the last section a nonlinear controller is developed using the Noninteracting approach, from Isidori, which is
introduced preliminarly. Then the results of the simulation in which the cube perform a pure yaw motion while balancing
on its corner are presented.

\newpage

\section{Equations of motion}
\begin{frame}{Configuration vector}
  \[  
  \vec{q} =
  \begin{bmatrix}
    \theta_{1} & \theta_{2} & \theta_{3} & q_{x} & q_{y} & q_{z}
  \end{bmatrix} \transpose
  \]
  \vskip-0.1in
  \begin{columns}
    \begin{column}{0.55\textwidth}
      \includegraphics[width=\columnwidth]{cubli.pdf}
    \end{column}
    \begin{column}{0.6\textwidth}
      \begin{block}{Attitude}
        \begin{itemize}
        \item[-] Body fixed reference frame $\{C\} = (O_c; \hat{\vec{i}}_c, \hat{\vec{j}}_c, \hat{\vec{k}}_c)$
        \item[-] Inertial reference frame $\{S\} = (O_s; \hat{\vec{i}}_s, \hat{\vec{j}}_s, \hat{\vec{k}}_s)$
        \item[-] $R_{SC}(\vec{\theta}) = R_z(\theta_1)R_y(\theta_2)R_x(\theta_3)$
        \end{itemize}
      \end{block}
      \begin{block}{Flywheels}
        $q_x$, $q_y$ and $q_z$ are the flywheel angular positions
      \end{block}
    \end{column}
  \end{columns}
\end{frame}

\begin{frame}{Notation \footnote[frame]{All quantities expressed in $\{C\}$}}
  In the follwing this notation will be used
  \begin{columns}
    \begin{column}{0.5\textwidth}
      \begin{block}{Velocities}
        \begin{itemize}
        \item[-] $\prescript{c}{}{\vec{v}}_{G_{c}}$: cubic frame CoM linear velocity
        \item[-] $\prescript{c}{}{\vec{v}}_{G_{i}}$: i-th flywheel CoM linear velocity
        \item[-] $\prescript{c}{}{\vec{\omega}}_{c}$: cubic frame angular velocity
        \item[-] $\prescript{c}{}{\vec{\omega}}_{i}$: i-th flywheel angular velocity
        \end{itemize}
      \end{block}
    \end{column}
    \begin{column}{0.5\textwidth}
      \begin{block}{Inertial quantities}
        \begin{itemize}
        \item[-] $m_c$: cubic frame mass
        \item[-] $m_i$: i-th flywheel mass
        \item[-] $\prescript{c}{}{\vec{G}}_{c}$: cubic frame CoM
        \item[-] $\prescript{c}{}{\vec{G}}_{i}$: i-th flywheel CoM 
        \item[-] $\prescript{c}{}{\mathfrak{J}}_{G_{c}}$: cubic frame inertia matrix
        \item[-] $\prescript{c}{}{\mathfrak{J}}_{G_{i}}$: i-th flywheel inertia matrix
        \end{itemize}
      \end{block}
    \end{column}
  \end{columns}
\end{frame}

\begin{frame}[shrink = 10]{Inertia matrices and parameters}
  \begin{columns}
    \begin{column}{0.55\textwidth}
      \begin{block}{Inertia matrices}
        \vskip-0.1in
        \[
        \begin{split}
          &\prescript{c}{}{\mathfrak{J}}_{G_{x}} = 
          \left[
            \begin{smallmatrix}
              \frac{m r^2}{2} & 0 & 0\\
              0 & \frac{m\left(h^2 + 3 r^2 \right)}{12}  & 0\\
              0 & 0 & \frac{m\left(h^2 + 3 r^2 \right)}{12}
            \end{smallmatrix}
            \right]\\
          &\prescript{c}{}{\mathfrak{J}}_{G_{y}} = 
          \left[
            \begin{smallmatrix}
              \frac{m\left(h^2 + 3 r^2 \right)}{12} & 0 & 0\\
              0 & \frac{m r^2}{2} & 0\\
              0 & 0 & \frac{m\left(h^2 + 3 r^2 \right)}{12}
            \end{smallmatrix}
            \right] \\
          &\prescript{c}{}{\mathfrak{J}}_{G_{z}} = 
          \left[
            \begin{smallmatrix}
              \frac{m\left(h^2 + 3 r^2 \right)}{12} & 0 & 0\\
              0 & \frac{m\left(h^2 + 3 r^2 \right)}{12} & 0\\
              0 & 0 & \frac{m r^2}{2}
            \end{smallmatrix}
            \right]
          \\
          &\prescript{c}{}{\mathfrak{J}}_{G_{c}} = 
          \left[
            \begin{smallmatrix}
              \frac{m_c a^2}{6} & 0 & 0\\
              0 & \frac{m_c a^2}{6} & 0\\
              0 & 0 & \frac{m_c a^2}{6}
            \end{smallmatrix}
            \right]
        \end{split}
        \] 
      \end{block}
    \end{column}
    \begin{column}{0.5\textwidth}
      \begin{block}{Parameters}
        \begin{itemize}
        \item[-] $M = \SI{2.5}{kg}$ (cubic frame mass)
        \item[-] $a = \SI{0.15}{m}$ (cube side)
        \item[-] $m = \SI{0.204}{kg}$ (flywheel mass)
        \item[-] $r = \SI{0.05}{m}$ (flywheel radius)
        \item[-] $h = \SI{0.005}{m}$ (flywheel height)
        \end{itemize}
      \end{block}
    \end{column}
  \end{columns}
\end{frame}

\begin{frame}[shrink=15]{Kinematics}
  \begin{columns}
    \begin{column}{0.5\textwidth}
      \[
      \begin{split}
        & \prescript{c}{}{\vec{\omega}}_c = 
        \begin{bmatrix}
          \prescript{c}{}{\tilde{J}}_{\omega} & 0_{3 \times 3}
        \end{bmatrix}\vec{\dot{q}} 
        = \prescript{c}{}{J}_{\omega} \vec{\dot{q}}\\
        & \prescript{c}{}{\vec{\omega}}_x = 
        \begin{bmatrix}
          \prescript{c}{}{\tilde{J}_{\omega}} & \vec{e}_1 & \vec{0} & \vec{0}
        \end{bmatrix} \dot{\vec{q}} = \prescript{c}{}{J}_{\omega_{x}} \dot{\vec{q}}\\
        &  \prescript{c}{}{\vec{\omega}}_y = 
        \begin{bmatrix}
          \prescript{c}{}{\tilde{J}_{\omega}} & \vec{0} & \vec{e}_2 & \vec{0}
        \end{bmatrix} \dot{\vec{q}} = \prescript{c}{}{J}_{\omega_{y}} \dot{\vec{q}}\\
        &  \prescript{c}{}{\vec{\omega}}_z = 
        \begin{bmatrix}
          \prescript{c}{}{\tilde{J}_{\omega}} & \vec{0} & \vec{0} & \vec{e}_3
        \end{bmatrix} \dot{\vec{q}} = \prescript{c}{}{J}_{\omega_{z}} \dot{\vec{q}}
      \end{split}
      \]
    \end{column}
    \begin{column}{0.5\textwidth}
      \[
      \prescript{c}{}{\tilde{J}}_{\omega} (\vec{q}) =
      \begin{bmatrix}
        -s_2 & 0 & 1 \\
        c_2 s_3 & c_3  & 0\\
        c_2 c_3 & -s_3 & 0
      \end{bmatrix}
      \]
    \end{column}
  \end{columns}
  \[
  \prescript{c}{}{\vec{v}}_{\vec{G}_{i}} =
  \cancelto{0}{\prescript{c}{}{\vec{v}}_{O_{c}}}_{\footnote[frame]{Spherical joint in $O_c = O_s$}} + \prescript{c}{}{\unitvec{\omega}}_c \prescript{c}{}{\vec{G}}_*
  = -\prescript{c}{}{\unitvec{G}}_* \prescript{c}{}{\vec{\omega}}_c
  = -\prescript{c}{}{\unitvec{G}}_* \prescript{c}{}{J_{\omega}} \dot{\vec{\theta}}
  = -\prescript{c}{}{\unitvec{G}}_* \prescript{c}{}{\tilde{J}_{\omega}} \dot{\vec{q}}
  \]
  \[
  \prescript{c}{}{\vec{G}}_{c} = 
  \begin{bmatrix}
    a/2 \\
    a/2 \\
    a/2
  \end{bmatrix} \quad
  \prescript{c}{}{\vec{G}}_x = 
  \begin{bmatrix}
    0 \\
    a/2 \\
    a/2
  \end{bmatrix} \quad
  \prescript{c}{}{\vec{G}}_y = 
  \begin{bmatrix}
    a/2 \\
    0 \\
    a/2
  \end{bmatrix} \quad
  \prescript{c}{}{\vec{G}}_z = 
  \begin{bmatrix}
    a/2 \\
    a/2 \\
    0
  \end{bmatrix}
  \]
\end{frame}

\begin{frame}[shrink=30]{Equation in standard manipulator form}
  \[
  B(\vec{q}) \ddot{\vec{q}} + C(\vec{q}, \dot{\vec{q}}) \dot{\vec{q}} + \vec{G}(\vec{q}) = \vec{Q}
  \]
  
  \begin{block}{Color code}
    \centering
    \cubicFrame{Cubic frame} $\quad$ \qx{Fly wheel x} $\quad$  \qy{Fly wheel y} $\quad$ \qz{Fly wheel z} 
  \end{block}
    \begin{block}{Joint space inertia matrix $B(\vec{q})$}
    \[
    \begin{split}
      B(\vec{q}) &= \cubicFrame{m_c \prescript{c}{}{J}_{\omega}\transpose \prescript{c}{}{\unitvec{G}}_{c} \transpose  \prescript{c}{}{\unitvec{G}}_{c} \prescript{c}{}{J}_{\omega}} + \cubicFrame{\prescript{c}{}{J}_{\omega}\transpose  \prescript{c}{}{\mathfrak{J}}_{G_{c}} \prescript{c}{}{J}_{\omega}} + 
      \qx{m_x \prescript{c}{}{J}_{\omega}\transpose \prescript{c}{}{\unitvec{G}}_{x} \transpose  \prescript{c}{}{\unitvec{G}}_{x} \prescript{c}{}{J}_{\omega}} + 
      \qx{\prescript{c}{}{J}_{\omega_{x}}\transpose  \prescript{c}{}{\mathfrak{J}}_{G_{x}} \prescript{c}{}{J}_{\omega_{x}}} \\
      &+ \qy{m_y \prescript{c}{}{J}_{\omega}\transpose \prescript{c}{}{\unitvec{G}}_{y} \transpose  \prescript{c}{}{\unitvec{G}}_{y} \prescript{c}{}{J}_{\omega}} + 
      \qy{\prescript{c}{}{J}_{\omega_{y}}\transpose  \prescript{c}{}{\mathfrak{J}}_{G_{y}} \prescript{c}{}{J}_{\omega_{y}}} 
      + \qz{m_z \prescript{c}{}{J}_{\omega}\transpose \prescript{c}{}{\unitvec{G}}_{z} \transpose  \prescript{c}{}{\unitvec{G}}_{z} \prescript{c}{}{J}_{\omega}} + \qz{\prescript{c}{}{J}_{\omega_{z}}\transpose  \prescript{c}{}{\mathfrak{J}}_{G_{z}} \prescript{c}{}{J}_{\omega_{z}}}\\
      & = B(\theta_2, \theta_3)
    \end{split}
    \]
  \end{block}
  \begin{columns}
    \begin{column}{0.5\textwidth}
      \begin{block}{Coriolis $C(\vec{q}, \dot{\vec{q}})$}
        \[
        C_{ij} = \sum_{k=1}^6{\Gamma^i_{jk} \dot{q}_k}
        \]
        \[
        \Gamma^i_{jk} = \frac{1}{2} \left ( \frac{\partial B_{ij}}{\partial q_k}
        + \frac{\partial B_{ik}}{\partial q_j}
        - \frac{\partial B_{jk}}{\partial q_i}\right)
        \]
      \end{block}
    \end{column}
    \begin{column}{0.5\textwidth}
      \begin{block}{Gravitational forces $G(\vec{q})$}
        \[
        \vec{G} =\left(\frac{\partial U}{\partial \vec{q}}\right) \transpose
        \]
        \[
        \begin{split}
          U(\vec{q}) = - \prescript{c}{}{\vec{g}} \transpose
          &(\cubicFrame{m_c  \prescript{c}{}{\vec{G}}_c} +
          \qx{m_x  \prescript{c}{}{\vec{G}}_x} \\
          &+\qy{m_y  \prescript{c}{}{\vec{G}}_y} +
          \qz{m_z  \prescript{c}{}{\vec{G}}_z})
        \end{split}
        \]
        \[
        \prescript{c}{}{\vec{g}} = R_{SC}\transpose \prescript{s}{}{\vec{g}} = R_{SC}\transpose 
        \begin{bmatrix}
          0 & 0 & -g
        \end{bmatrix} \transpose
        \]
      \end{block}
    \end{column}
  \end{columns}
\end{frame}

\begin{frame}{Derivation of the generalized forces $\vec{Q}$}
  $\vec{\tau} = \begin{bmatrix}
    \tau_x & \tau_y &\tau_z
  \end{bmatrix} \transpose$ are the torques applied to the flywheels
  \[
  \begin{split}
    Q_i = & \tau_{x} \hat{\vec{i}}_c^{T} \frac{\partial (\prescript{c}{}{\vec{\omega}_x})}{\partial \dot{q}_i} +
    \tau_{y} \hat{\vec{j}}_c^{T} \frac{\partial (\prescript{c}{}{\vec{\omega}_y})}{\partial \dot{q}_i} +
    \tau_{z} \hat{\vec{k}}_c^{T} \frac{\partial (\prescript{c}{}{\vec{\omega}_z})}{\partial \dot{q}_i} -\\
    &\tau_{x} \hat{\vec{i}}_c^{T} \frac{\partial (\prescript{c}{}{\vec{\omega}_c})}{\partial \dot{q}_i} -
    \tau_{y} \hat{\vec{j}}_c^{T} \frac{\partial (\prescript{c}{}{\vec{\omega}_c})}{\partial \dot{q}_i} -
    \tau_{z} \hat{\vec{k}}_c^{T} \frac{\partial (\prescript{c}{}{\vec{\omega}_c})}{\partial \dot{q}_i}
  \end{split}
  \]
  \[
  \vec{Q} =
  \begin{bmatrix}
    0 \\ 0 \\ 0 \\ \vec{\tau}
  \end{bmatrix}
  = F \vec{\tau} = 
  \begin{bmatrix}
    0_{3\times3} \\
    I_{3\times3}
  \end{bmatrix} \vec{\tau}
  \]
  \[ 
  \text{rank}(F) = 3 < 6 \quad \forall \vec{q} \implies \text{underactuated system}
  \]
\end{frame}

\begin{frame}{Equations in control affine form}
  \begin{block}{State space vector}
    \[
    \tilde{\vec{x}} =
    \begin{bmatrix}
      \tilde{\vec{x}}_1\transpose & \tilde{\vec{x}}_2\transpose
    \end{bmatrix}\transpose
    \]
    \[
    \tilde{\vec{x}}_1 = 
    \begin{bmatrix}
      \theta_1 & \theta_2 & \theta_3 & q_x & q_y & q_z
    \end{bmatrix}
    \enspace
    \tilde{\vec{x}}_2 = 
    \begin{bmatrix}
      \dot{\theta}_1 & \dot{\theta}_2 & \dot{\theta}_3 & \dot{q}_x & \dot{q}_y & \dot{q}_z
    \end{bmatrix}
    \]
  \end{block}
  \[
  \begin{split}
    \dot{\tilde{\vec{x}}} &= 
    \begin{bmatrix}
      \tilde{\vec{x}}_2 \\
      - B({\tilde{\vec{x}}_1}) ^ {-1} (C(\tilde{\vec{x}}_1, \tilde{\vec{x}}_2) \tilde{\vec{x}}_2 + \vec{G}(\tilde{\vec{x}}_1)) 
    \end{bmatrix} +
    \begin{bmatrix}
      \vec{0} \\
      B({\tilde{\vec{x}}_1}) ^ {-1} \vec{Q}
    \end{bmatrix}\\
    &=\begin{bmatrix}
    \tilde{\vec{x}}_2 \\
    - B^ {-1} (C \tilde{\vec{x}}_2 + \vec{G}) 
    \end{bmatrix} +
    \begin{bmatrix}
      0_{6\times3} \\
      \left(B({\tilde{\vec{x}}_1}) ^ {-1}\right)_{(:, 4:6)}
    \end{bmatrix}\vec{\tau}\\
    &= \tilde{\vec{f}}(\tilde{\vec{x}}) + \tilde{g}(\tilde{\vec{x}}) \vec{\tau}
    = \tilde{\vec{f}}(\tilde{\vec{x}}) + \tilde{g}(\tilde{\vec{x}}) \vec{u}
  \end{split}
  \]
\end{frame}

\begin{frame}{Equations in control affine form (continued)}
  $B$, $C$ and $\vec{G}$ do \alert{not} depend on $q_x$, $q_y$ and $q_z$ and their dynamics can be neglected
  \[
  \vec{x} =
  \begin{bmatrix}
    \theta_1 & \theta_2 & \theta_3 & \dot{\theta}_1 & \dot{\theta}_2 & \dot{\theta}_3 & \dot{q}_x & \dot{q}_y & \dot{q}_z
  \end{bmatrix}\transpose = \begin{bmatrix}
    \vec{\theta} & \vec{\dot{\theta}} & \vec{\dot{q}}_{wheels}
  \end{bmatrix}\transpose
  \]
  \[
  \begin{cases}
    \dot{\vec{x}} = \vec{f}(\vec{x}) + g(\vec{x}) \vec{u} \\
    \vec{y} = \vec{h}(\vec{x}) =
    \begin{bmatrix}
      \theta_1 &
      \theta_2 &
      \theta_3
    \end{bmatrix} \transpose
  \end{cases}
  \]
  where
  \[
  g(\vec{x}) = \left(B(\alert{\theta_2,\theta_3}) ^ {-1}\right)_{(:, 4:6)}
  \]
  and $\vec{f}(\vec{x})$ do \alert{not} depend on $\theta_1$

\end{frame}

\begin{frame}[shrink = 30]{Equlibrium points}
  \begin{center}
    \includegraphics[scale=0.4]{cubli_equilibria}
  \end{center}
  \begin{block}{Upright equilibria (unstable)}
    \[
    \mathcal{E}_{1} = \left\{ \vec{x} \mid \theta_1 \in [-\pi, \pi),\enspace \theta_2 = -\mathrm{atan}\frac{\sqrt{2}}{2},\enspace
      \theta_3 = \frac{\pi}{4},\enspace
      \vec{\dot{\theta}} = \vec{0},\enspace
      \vec{\dot{q}}_{wheels} = \vec{0},\enspace
      \vec{u} = \vec{0}
      \right\}
      \]
  \end{block}
  \begin{block}{Hanging equilibria (stable)}
    \[
    \mathcal{E}_{2} = \left\{
    \vec{x} \mid \theta_1 \in [-\pi, \pi),\enspace
      \theta_2 = \mathrm{atan}\frac{\sqrt{2}}{2},\enspace
      \theta_3 = -\frac{3}{4} \pi,\enspace
      \vec{\dot{\theta}} = \vec{0},\enspace
      \vec{\dot{q}}_{wheels} = \vec{0},\enspace
      \vec{u} = \vec{0}
      \right\}
      \]
  \end{block}
\end{frame}

\section{Controllability}

\begin{frame}{Small time local controllability \theory}
  Given a point $\vec{x}_0$ and an arbitrarily small neighbourhood $V(\vec{x}_0)$ define the set
  \[
  R_{T}^{V}(\vec{x}_0) = \{\vec{x}(\vec{x_0}, T, \bar{\vec{u}}(t)) \mid
  \vec{x}(\vec{x_0}, \tau, \bar{\vec{u}}(t)) \in V(\vec{x}_0) \enspace \forall \tau \in [0,T]\}
  \]
  with $\bar{\vec{u}}$ appropriate inputs
  \par
  A nonlinear system $\dot{\vec{x}} = \vec{f}(\vec{x}, \vec{u})$ is said \alert{locally-locally controllable} (l.l.c.) from $\vec{x}_0$ if
  \[
  \forall \enspace V(\vec{x}_0) \enspace \exists  \enspace T \enspace
  s.t. \enspace R_{T}^{V}(\vec{x}_0) \enspace \supseteq \enspace B(\vec{x}_0)
  \]
  with $B(\vec{x}_0)$ an open neighbourhood
  \par
  If the time $T$ can be arbitrarily small the system is said \alert{small time locally controllable} (s.t.l.c.)
\end{frame}

\begin{frame}{s.t.l.c and linear controllability \theory}
  \begin{theorem}
    Consider a nonlinear system $\dot{\vec{x}} = \vec{f}(\vec{x}, \vec{u})$ and an equilibrium point $\vec{x}_{eq}$
    such that $\vec{f}(\vec{x}_{eq}, \vec{0}) = \vec{0}$. If the linear approximation of the system around
    the point $\vec{x}_{eq}$ is completely controllable then the system is small-time locally controllable
    in $\vec{x}_{eq}$.
  \end{theorem}
\end{frame}

\begin{frame}{s.t.l.c and linear controllability \cubli}
  \begin{exampleblock}{Controllability of the linear approximation of the Cubli}
  \[
  \begin{split}
    \dot{\vec{\delta x}} &= \frac{\partial{\vec{f}(\vec{x},\vec{u})}}{\partial{\vec{x}}}\Big|_{\mathcal{E}_1} \vec{\delta x} +
    \frac{\partial{\vec{f}(\vec{x},\vec{u})}}{\partial{\vec{u}}}\Big|_{\mathcal{E}_1} \vec{\delta u}\\
    &= A \vec{\delta x} + B \vec{\delta u}
  \end{split}
  \]
  where $\vec{f}(\vec{x},\vec{u})$ does \alert{not} depend on $\theta_1$
  \par
  The pair $(A, B)$ is \alert{not} completely controllable. Indeed
  \[
  rank\left(
  \begin{bmatrix}
    B & AB & A^2B & \dots & A^{n-1}B
  \end{bmatrix}
  \right) = 8 < n = 9
  \]
  \end{exampleblock}
  It is \alert{not possible to conclude} on the small-time local controllability of the system
  in those points belonging to $\mathcal{E}_1$
\end{frame}

\begin{frame}{Small time local accessibility \theory}
  If $R_{T}^{V}(\vec{x}_0)$ only contains an open set of $\vec{x}_0$
  the small time local controllability becomes \alert{small-time local accessibility}
  (s.t.l.a).
  \par
  \[
  \Rightarrow
  \]
  \[
  \text{s.t.l.c} \qquad \text{s.t.l.a}
  \]
  \[
  \centernot \Leftarrow
  \]
\end{frame}

\begin{frame}[shrink=10]{Small time local accessibility (continued) \theory}
  Consider a control affine system
  \[
  \dot{\vec{x}} = \vec{f}(\vec{x}) + \sum\limits_{j=1}^{m}\vec{g}_j(\vec{x}) u_{j}
  \]
  and the distributions
  \[
  \mathrm{\Delta}_0 = \mathrm{span} \{\vec{g}_1, \enspace \hdots \enspace, \vec{g}_m\} \quad
  \quad
  \mathrm{\Delta} = \mathrm{span} \{\vec{f},\enspace\vec{g}_1, \enspace \hdots \enspace, \vec{g}_m\}
  \]
  \begin{theorem}
    Consider the smallest $\mathrm{\Delta}$-invariant distribution containing $\mathrm{\Delta}_0$
    called \alert{accessibility distribution} $<\mathrm{\Delta}|\mathrm{\Delta}_0>$.
    \begin{itemize}
    \item[a.]If the dimension of $<\mathrm{\Delta}|\mathrm{\Delta}_0>= n$ in $\vec{x}_0$ then system
      is \alert{s.t.l.a in $\vec{x}_{0}$};
    \item[b.]If $\mathrm{dim}(<\mathrm{\Delta}|\mathrm{\Delta}_0>)= r < n$ in a neighbourhood of $\vec{x}_0$
      then the set $R_{T}^{V}(\vec{x}_0)$ is contained in a submanifold of dimension $r$ of the
      $n$-dimensional state space, and contains an open set in that submanifold.
    \end{itemize}
  \end{theorem}
\end{frame}

\begin{frame}{Accessibility distribution \theory}
  In order to evaluate $<\mathrm{\Delta}|\mathrm{\Delta}_0>$
  the following filtration of distributions
  \[
  \begin{cases}
    \mathrm{\Delta}_1 &= \mathrm{\Delta}_0 + [\mathrm{\Delta}_0,\mathrm{\Delta}]\\
    &\vdotswithin{=} \\
    \mathrm{\Delta}_k &= \mathrm{\Delta}_{k-1} + [\mathrm{\Delta}_{k-1},\mathrm{\Delta}]
  \end{cases}
  \]
  is performed until an integer $k$ is found s.t.
  \[
  \text{dim}(\mathrm{\Delta}_{k}(\vec{x}_0)) = \text{dim}(\mathrm{\Delta}_{k+1}(\vec{x}_0))
  \]
\end{frame}

\begin{frame}{Accessibility distribution \cubli}
  \begin{exampleblock}{Accessibility distribution of the Cubli}
    Using the Symbolic Math Toolbox from MATLAB it turns out that for all $\vec{x}_{0} \in \mathcal{E}_{1}$ 
    \[
    \begin{split}
      &k = 3\\
      &\mathrm{dim}(<\mathrm{\Delta}|\mathrm{\Delta}_0>) = 8 < n = 9
    \end{split}
    \]
    Hence $R_{T}^{V}(\mathcal{E}_{1})$ is containted in a submanifold of dimension $8$
  \end{exampleblock}
\end{frame}

\begin{frame}{Weak local accessibility \theory}
  Although the system is not s.t.l.a in the points of interest it could be local accessible
  in a weaker sense, i.e., without the requirement that the time $T$ is arbitrarily small.
  \begin{theorem}
    A \alert{necessary} condition for weak local accessibility is
    \[
    \mathrm{dim}(<\mathrm{\Delta}|\mathrm{\Delta}>) = n
    \]
  \end{theorem}
\end{frame}

\begin{frame}{Weak local accessibility \cubli}
  \begin{exampleblock}{Weak local accessibility of the Cubli}
    Using the Symbolic Math Toolbox from MATLAB it turns out that
    \[
    \mathrm{dim} <\mathrm{\Delta} | \mathrm{\Delta}> = 8 < n = 9
    \quad \forall \vec{x}_0 \in \mathcal{E}_1
    \]
  \end{exampleblock}
  \alert{The system is not locally accessible in any sense hence it can't be locally controllable}
\end{frame}

\section{Observability}

\begin{frame}{Local observability and indistinguishability}
The concept of local observability can be explained in terms of the indistinguishability
between two ``near'' \emph{initial} states $\bar{\vec{x}}$ and $\bar{\vec{x}} + \vec{\delta x}$.
\par
The initial states $\bar{\vec{x}}$ and $\bar{\vec{x}} + \vec{\delta x}$ are said to be
indistinguishable in the interval $[0, T]$ if
\[
\vec{y}(\bar{\vec{x}}, \vec{u}, t) = \vec{y}(\bar{\vec{x}} + \vec{\delta x}, \vec{u}, t)\enspace
\forall \vec{u} \in U \enspace \forall t \in [0,T]
\]
where $U \subseteq \{\vec{u}(\cdot):[0,T] \rightarrow \mathbb{R}^{m}\}$



\end{frame}

\section{Noninteracting controller}

\begin{frame}[shrink=10]{Theory of Noniteracting control}
  Consider a nonlinear control affine system with $m$ inputs and $m$ outputs
  \[
  \vec{\dot{x}} = \vec{f}(\vec{x}) + \sum\limits_{i=1}^{\alert{m}} \vec{g}_{i}(\vec{x}) u_{i}
  \]
  \[
  y_{1} = h_{1}(\vec{x})
  \]
  \[
  \hdots
  \]
  \[
  y_{m} = h_{\alert{m}}(\vec{x})
  \]
  and an initial point $\vec{x}^{0}$
  \par
  The problem of noninteracting control is stated as finding a regular static state feedback
  \[
  \vec{u}(\vec{x})  = \vec{\alpha}(\vec{x}) + \vec{\beta}(\vec{x}) \vec{v} \quad \vec{x} \in B_{\delta}(\vec{x}^0)
  \]
  such that in the resulting closed loop system each output $y_i$ is affected
  only by the corresponding input $v_i$ and not by the others.
\end{frame}

\begin{frame}{Theory of Noniteracting control (continued)}
  Suppose that the system has a vector relative degree $\{r_1, \hdots, r_m\}$ at $\vec{x}^{0}$ hence it admits
  a \emph{normal form} locally around $\vec{x}^{0}$
  \begin{columns}[t]
    \begin{column}{0.5\textwidth}
      \[
      \dot{\xi}_{1}^{i} = \xi_{2}^{i}
      \]
      \[
      \hdots
      \]
      \[
      \dot{\xi}_{r_{i}-1}^{i} = \xi_{r_{i}}^{i}
      \]
      \[
      1 \le i \le m
      \]
      \[
      \begin{bmatrix}
        \dot{\xi}_{r_{1}}^{1}\\
        \vdots\\
        \dot{\xi}_{r_{m}}^{m}\\
      \end{bmatrix}=
      \vec{b}(\vec{\xi},\vec{\eta}) + A(\vec{\xi},\vec{\eta})\vec{u}
      \]
    \end{column}
    \begin{column}{0.5\textwidth}
      \[
      \vec{\dot{\eta}} = \vec{q}(\vec{\xi},\vec{\eta}) + \vec{p}(\vec{\xi},\vec{\eta})\vec{u}
      \]
      \vskip0.3in
      \[
      \begin{bmatrix}
        \vec{\xi}\\
        \vec{\eta}
      \end{bmatrix} = 
      \Phi(\vec{x})
      \]
    \end{column}
  \end{columns}
  \centering
  with $A(\vec{x})$ nonsingular at $\vec{x}^{0}$
\end{frame}

\begin{frame}{Theory of Noniteracting control (continued)}
  \begin{block}{Proposition (Isidori)}
    Suppose
    \[
    L{\vec{g}_j}L_{\vec{f}}^{k}h_{i}(\vec{x}) = 0 \enspace \forall \vec{x} \in B_{\delta}(\vec{x}^0)
    \]
    \[
    1 \le j \le m, \enspace 1 \le i \le m, \enspace k < r_i-1
    \]
    and
    \[
    \begin{bmatrix}
      L_{\vec{g}_1}L_{\vec{f}}^{r_{i-1}}h_i(\vec{x}^{0}) \enspace \hdots \enspace L_{\vec{g}_m}L_{\vec{f}}^{r_i-1}h_i(\vec{x}^{0})
    \end{bmatrix}
    \ne
    \begin{bmatrix}
      0 \enspace \hdots \enspace 0
    \end{bmatrix}
    \enspace 1 \le i \le m
    \]
    Then the noninteracting control problem is solvable iff the matrix $A(\vec{x}^{0})$ is nonsingular,
    i.e. if the system has a vector relative degree $\{r_1, \hdots, r_m\}$ at $\vec{x}^{0}$
  \end{block}
\end{frame}

\begin{frame}{Theory of Noniteracting control (continued)}
  Indeed the regular static feedback
  \[
  \vec{u} = -A^{-1}(\vec{\xi},\vec{\eta})\vec{b}(\vec{\xi},\vec{\eta}) + A^{-1}(\vec{\xi},\vec{\eta})\vec{v}
  \]
  results in the closed loop system
  \begin{columns}[t]
    \begin{column}{0.5\textwidth}
      \[
      \dot{\xi}_{1}^{i} = \xi_{2}^{i}
      \]
      \[
      \hdots
      \]
      \[
      \dot{\xi}_{r_{i}-1}^{i} = \xi_{r_{i}}^{i}
      \]
      \[
      \dot{\xi}_{r_{i}}^{i} = v_{i}
      \]
      \[
      y_{i} = \xi_{1}^{i}
      \]
      \[
      1 \le i \le m
      \]
    \end{column}
    \begin{column}{0.5\textwidth}
      \[
      \vec{\dot{\eta}}  = \hat{\vec{q}}(\vec{\xi},\vec{\eta}) + \hat{\vec{p}}(\vec{\xi},\vec{\eta})\vec{v}
      \]
    \end{column}
  \end{columns}
\end{frame}

\begin{frame}{Application of the Noniteracting control}
  For the system under examination
  \[
  \begin{cases}
    \begin{split}
      \dot{\vec{x}} &= \vec{f}(\vec{x}) + g(\vec{x})\vec{u}\\
      &=\vec{f}(\vec{x}) +
      \begin{bmatrix}
        0_{3\times3} \\
        \left(B(\theta_2,\theta_3) ^ {-1}\right)_{(:, 4:6)}
      \end{bmatrix}\vec{u}\\
    \end{split}\\
    \vec{y} = \vec{h}(\vec{x}) =
    \begin{bmatrix}
      \theta_1 &
      \theta_2 &
      \theta_3
    \end{bmatrix} \transpose
  \end{cases}
  \]
  it can be found that
  \[
  L_{\vec{g}_{j}}L_{\vec{f}}^{0}h_{i} =
  L_{\vec{g}_{j}}h_{i} = 
  \begin{bmatrix}
    \vec{e}_i & 0_{6x1}
  \end{bmatrix}
  \vec{g}_j
  =
  g_{j}^{i} = 0
  \qquad 1 \le j \le 3, \enspace 1 \le i \le 3
  \]

\end{frame}

\begin{frame}{Existence of a vector relative degree}
  Also 
  \[
  \frac{\mathrm{d}^{2}}{\mathrm{d}t^{2}}\vec{y} = 
  \begin{bmatrix}
    \ddot{\theta_1}\\
    \ddot{\theta_2}\\
    \ddot{\theta_3}
  \end{bmatrix} =
  \vec{b}(\vec{x}) + A(\vec{x}) \vec{u}
  \]
  where
  \[
  \vec{b}(\vec{x}) = \vec{f}(\vec{x})_{(4:6)}
  \]
  and
  \[
  A(\vec{x}) = \left(B(\theta_2,\theta_3) ^ {-1}\right)_{(1:3, 4:6)}
  \]
\end{frame}

\begin{frame}{Existence of a vector relative degree (continued)}
  \centering
  \includegraphics[scale=0.4]{detA.png}
  \[
  |A| \ne 0 \enspace \forall \theta_2, \theta_3 \implies \{r_1, r_2, r_3\} = \{2,2,2\} \enspace \forall \vec{x}
  \]
  The resulting standard noninteractive feedback is
  \[
  \vec{u}_{NIC} = -\left(B(\theta_2,\theta_3) ^ {-1}\right)_{(1:3, 4:6)}^{-1}
  (\vec{f}_{(4:6)}(\vec{x}) - \vec{v} )
  \]
\end{frame}

\begin{frame}[shrink=10]{Unobservable part of the system}
  The unobservable part of the system can be obtained by completely specifying
  the coordinate transformation $\Phi$
  \[
  \vec{y} =
  \begin{bmatrix}
    \theta_1 &
    \theta_2 &
    \theta_3
  \end{bmatrix}\transpose
  \implies
  \Phi(\vec{x})_{1:6} =
  \begin{bmatrix}
    \theta_1 & \dot{\theta}_1 &  \theta_2 & \dot{\theta}_2 &
    \theta_3 & \dot{\theta}_3
  \end{bmatrix}\tranpose
  \]
  The remaining part is chosen as
  \[
  \Phi(\vec{x})_{1:6} =
  \begin{bmatrix}
    \dot{q}_x & \dot{q}_y & \dot{q}_z
  \end{bmatrix}\transpose
  \]
  i.e. the unobservable part of the closed loop system is the closed loop dynamics of the
  flywheels
  \[
  \begin{split}
    \dot{\vec{\eta}} &=
    \begin{bmatrix}
      \ddot{q}_{x}\\
      \ddot{q}_{y}\\
      \ddot{q}_{z}
    \end{bmatrix}
    = \vec{f}_{(7:9)}(\vec{x}) + \left(B(\theta_2,\theta_3) ^ {-1}\right)_{(4:6, 4:6)} \vec{u}_{NIC}\\
    &=\vec{f}_{(7:9)}(\vec{x}) - \left(B(\theta_2,\theta_3) ^ {-1}\right)_{(4:6, 4:6)}
    \left(B(\theta_2,\theta_3) ^ {-1}\right)_{(1:3, 4:6)}^{-1}
    (\vec{f}_{(4:6)}(\vec{x}) - \vec{v})\\
    &=\hat{\vec{q}}(\vec{x}) + \hat{\vec{p}}(\theta_{2},\theta_{3})\vec{v}
  \end{split}
  \]
\end{frame}

\begin{frame}{Yaw motion while balancing on a corner}
  Consider a rotation about the $z$ axis of the inertial
  reference frame $\{S\}$ while the balance on one of the corner is maintained
  \begin{center}
    \includemedia[
      activate=pageopen,
      width=90pt,height=90pt,
      addresource=videos/cubli_yaw_noerr.mp4,
      flashvars={%
        src=videos/cubli_yaw_noerr.mp4
        &scaleMode=stretch&autoPlay=true&loop=true
        &hideBar=true}
    ]{}{StrobeMediaPlayback.swf}
  \end{center}
\end{frame}

\begin{frame}{Yaw motion while balancing on a corner (continued)}
  A suitable control law $\vec{v}$ is
  {\small
    \[
    \begin{split}
      &v_1(t) = \ddot{\theta}_1^{des}(t) + k_d(\dot{\theta}_1^{des}(t) - \dot{\theta}_1(t)) + k_p (\theta_1^{des}(t) - \theta_1(t))\\
      &v_2(t) = - k_d\dot{\theta}_2(t) + k_p \left(-\mathrm{atan}\frac{\sqrt{2}}{2} - \theta_2(t)\right)\\
      &v_3(t) = - k_d\dot{\theta}_3(t) + k_p \left(\frac{\pi}{4} - \theta_3(t)\right)
    \end{split}
    \]
  }
  where
  \[
  \theta_1^{des}(t) = a_0 + a_1 t + a_2 t^2 + a_3 t^3 + a_4 t^4 + a_5 t^5
  \]
  with boundary conditions
  \[
  \begin{split}
    &\theta_1^{des}(0) \in [-\pi, \pi) \quad \dot{\theta}_1^{des}(0)= 0 \quad \ddot{\theta}_1^{des}(0) = 0\\
      &\theta_1^{des}(t_{f}) \in [-\pi, \pi)  \quad \dot{\theta}_1^{des}(t_{f})= 0 \quad \ddot{\theta}_1^{des}(t_{f}) = 0
  \end{split}
  \]
\end{frame}

\begin{frame}{Results}
  \[\theta_1^{des} = \SI{0}{\degree} \rightarrow \theta_1^{des}(2) = \SI{100}{\degree}
  \rightarrow \theta_1^{des}(4) = \SI{0}{\degree}\]
  \[k_p = 10 \enspace k_d = 0.1\]
  \begin{center}
    \includemedia[
      activate=pageopen,
      width=288pt,height=162pt,
      addresource=videos/cubli_yaw.mp4,
      flashvars={%
        src=videos/cubli_yaw.mp4
        &scaleMode=stretch&autoPlay=true}
    ]{}{StrobeMediaPlayback.swf}
  \end{center}
\end{frame}

\begin{frame}{Results (continued)}
  The flywheel velocities $\vec{\eta}(t)$ which are
  solution to the differential equation
  \[
  \begin{cases}
    \dot{\vec{\eta}} = \hat{\vec{q}}(\vec{x}) + \hat{\vec{p}}\left(-\mathrm{atan}\frac{\sqrt{2}}{2},\frac{\pi}{4}\right)\vec{v}_{PD}\\
    \vec{\eta}(0) = \vec{0}
  \end{cases}
  \]
  are bounded and tends to zero
  \par
  \centering
  \includegraphics[scale=0.4]{fly_wheel}
\end{frame}

\begin{frame}{Use of an alternative Euler parametrization}
\end{frame}

\begin{frame}{Results}
  \vskip0.1in
  \begin{columns}
    \begin{column}{0.45\textwidth}
      \begin{itemize}
      \item[-]$k_p = 2500$, $k_d = 50$
      \item[-]rotation of $\SI{180}{\degree}$ in $\SI{4}{\second}$
      \item[-]LQR control actived at $t = \SI{5}{\second}$ 
      \end{itemize}
    \end{column}
    \begin{column}{0.45\textwidth}
      \includegraphics[width=\columnwidth]{error_lqr}
    \end{column}
  \end{columns}
  \begin{columns}
    \begin{column}{0.45\textwidth}
      \includegraphics[width=\columnwidth]{fly_wheel_lqr}
    \end{column}
    \begin{column}{0.45\textwidth}
      \includegraphics[width=\columnwidth]{input_lqr}
    \end{column}
  \end{columns}
\end{frame}

%% \begin{frame}{Results ($k_p = 2500$, $k_d = 50$)}
%%   \centering
%%   \includegraphics[scale=0.4]{error_lqr}
%%   \includegraphics[scale=0.4]{input_lqr}
%% \end{frame}

%% \begin{frame}{Results}
%%   The LQR control is actived at $t = \SI{5}{\second}$
%%   \par
%%   \centering
%%   \includegraphics[scale=0.6]{fly_wheel_lqr}
%% \end{frame}

\begin{frame}{Results}
  The outcome of the switch to the LQR controller
  is shown for several angles of rotation and several durations of the movement
  \par
  \centering
  \includegraphics[scale=0.46]{simulation_lqr}
\end{frame}

\begin{frame}{Results}
  The failures happen when the flywheel velocities are too high as expected
  \par
  \centering
  \includegraphics[scale=0.62]{simulation_lqr_velocities}
\end{frame}



\end{document}

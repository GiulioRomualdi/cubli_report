\section{Controllability}
In this section the nonlinear \emph{local} controllability of the system is discussed.
\par
A nonlinear system is locally controllable in a point $\vec{x}_0$ if there exists
a neighbourhood $B_\epsilon (\vec{x}_0)$ such that for every point $\vec{x}_f$
in that neighbourhood a control law $\vec{u}(t)$ exists such that  $\vec{x}(T, \vec{x}_0, \vec{u}) = \vec{x}_f$.
If $T$ can be arbitrarily small then the system is defined as locally
controllable in small-time from $\vec{x}_{0}$.
\par
Local controllability can be assessed using tools taken from  differential geometry of manifolds
and Lie algebras. However LTI theory can also be used in some special cases that are described
in the following.

\subsection{Linear controllability}
In order to study the local controllability of nonlinear systems using the theory of LTI systems
the following theorem can be used.
\begin{theorem}
  Consider a nonlinear system $\dot{\vec{x}} = \vec{f}(\vec{x}, \vec{u})$ and an equilibrium point $\vec{x}_{eq}$
  such that $\vec{f}(\vec{x}_{eq}, \vec{0}) = \vec{0}$. If the linear approximation of the system around
  the point $\vec{x}_{eq}$ is completely controllable then the system is small-time locally controllable
  in $\vec{x}_{eq}$.
\end{theorem}
It is recalled that a linear system
\[
\dot{\vec{\delta x}} = A \vec{\delta x} + B \vec{\delta u}
\]
where
\[
\vec{\delta x}(t) = \vec{x}(t) - \vec{x}_{eq} \quad \vec{\delta u}(t) = \vec{u}(t) - \vec{u}_{eq}
\]
is completely controllable if the rank of the controllability matrix $R_c$ 
\begin{equation}
R_c = 
\begin{bmatrix}
B & AB & A^2B & \dots & A^{n-1}B
\end{bmatrix}
\end{equation}
is equal to the dimension of the state $n$.
\par
For the system of equations (\ref{eq:nl_eq_2}) it can be found that the linear approximation
around each equilibrium point belonging to $\mathcal{E}_{1}$ is not completely controllable.
Indeed the controllability matrix has rank $8 < n = 9$ for every $\vec{x}_{eq} \in \mathcal{E}_1$
hence it is not possible to conclude on the small-time local controllability of the system
in those points.
\par
As an aside is interesting to study the stabilizability of the linear approximation.
Using the Control System Toolbox from MATLAB a change of coordinates can be found such that
\[
\dot{\vec{\delta z}} = \bar{A} \vec{\delta z} + \bar{B} \vec{\delta u}
\]
where
\[
\bar{A} =TAT\transpose = 
\begin{bmatrix}
  A_{uc} & 0\\
  A_{21} & A_c
\end{bmatrix}
\]
\[
\bar{B} =TB = 
\begin{bmatrix}
0\\
B_c
\end{bmatrix}
\]
and $T$ is an unitary matrix. %% For the system under investigation
%% $A_{uc} = 0$, i.e., the uncontrollable eigenvalue is $0$ with algebraic multiplicity one
%% hence the system is stabilizable with a control law of the form $\vec{\delta u} = K \vec{\delta x}$
%% for some appropriate constant matrix $K$. Using arguments based on the theory of Lyapunov stability
%% it can be also conluded that the nonlinear system of equations (\ref{eq:nl_eq_2}) subjected to the same
%% control law has a \emph{stable} point of equilibrium at $\vec{x}_{eq} \in \mathcal{E}_{1}$.
For the system under investigation $A_{uc} = 0$, i.e., the uncontrollable eigenvalue is $0$ with algebraic
multiplicity one hence the uncontrollable part of the system is not asymptotically stable and the system
is not stabilizable. However is still possible to find control law of the form $\vec{\delta u} = K \vec{\delta x}$
which asymptotically stabilize the controllable part of the system resulting in a overall neutrally stable
closed loop system.
\subsection{Nonlinear local accessibility}
Local controllability of nonlinear systems is a quite difficult property to be verified.
For this reason a different definition of local controllability is used in which,
in addition, is required that the trajectory from the initial point $\vec{x}_0$ to the
final point $\vec{x}(T)$ never goes outside of a small neighbourhood $V(\vec{x}_0)$.
To be more precise a set $R_{T}^{V}(\vec{x}_0)$ is defined
\[
R_{T}^{V}(\vec{x}_0) = \{\vec{x}(\vec{x_0}, T, \bar{\vec{u}}(t)) \mid
\vec{x}(\vec{x_0}, \tau, \bar{\vec{u}}(t)) \in V(\vec{x}_0) \enspace \forall \tau \in [0,T]\}
\]
Then a nonlinear system is said locally-locally controllable (l.l.c.) from $\vec{x}_0$ if,
for any arbitrarily small neighbourhood $V(\vec{x}_0)$ there exists $T$ such that
$R_{T}^{V}(\vec{x}_0)$ contains an open neighbourhood of $\vec{x}_0$.
Again if the time $T$ can be arbitrarily small the system is said
small time locally controllable (s.t.l.c.).
\par
In some cases however $R_{T}^{V}(\vec{x}_0)$ only contains an open set of $\vec{x}_0$
and the definition of small-time local controllability becomes small-time local accessibility
(s.t.l.a). It should be noted that s.t.l.c implies s.t.l.a but the converse does not hold.
\par
In order to study the small-time locally accessibility of a system written in control affine form
\begin{equation}\label{eq:nonlinear_system}
\dot{\vec{x}} = \vec{f}(\vec{x}) + g(\vec{x}) \vec{u}
\end{equation}
the distributions
\[
\mathrm{\Delta}_0 = \mathrm{span} \{\vec{g}_1, \enspace \hdots \enspace, \vec{g}_m\} \quad
\]
\[
\mathrm{\Delta} = \mathrm{span} \{\vec{f},\enspace\vec{g}_1, \enspace \hdots \enspace, \vec{g}_m\}
\]
have to be considered. The following theorem from Chow holds
\begin{theorem}
  Consider the smallest $\mathrm{\Delta}$-invariant distribution containing $\mathrm{\Delta}_0$ called
  accessibility distribution $<\mathrm{\Delta}|\mathrm{\Delta}_0>$.
  \begin{itemize}
  \item[a.]If the dimension of $<\mathrm{\Delta}|\mathrm{\Delta}_0>= n$ in $\vec{x}_0$, then system (\ref{eq:nonlinear_system})
    is s.t.l.a in $\vec{x}_{0}$;
  \item[b.]If $\mathrm{dim}(<\mathrm{\Delta}|\mathrm{\Delta}_0>)= r < n$ in a neighbourhood of $\vec{x}_0$
    then the set $R_{T}^{V}(\vec{x}_0)$ is contained in a submanifold of dimension $r$ of the
    $n$-dimensional state space, and contains an open set in that submanifold.
  \end{itemize}
\end{theorem}
In order to evaluate the accessibility distribution for the system under examination
(equation \ref{eq:nl_eq_2}) the following filtration of distributions
\[
\begin{cases}
\mathrm{\Delta}_1 &= \mathrm{\Delta}_0 + [\mathrm{\Delta}_0,\mathrm{\Delta}]\\
&\vdotswithin{=} \\
\mathrm{\Delta}_k &= \mathrm{\Delta}_{k-1} + [\mathrm{\Delta}_{k-1},\mathrm{\Delta}]
\end{cases}
\]
was performed until an integer $k$ was found such that
\[
\text{dim}(\mathrm{\Delta}_{k}(\vec{x}_0)) = \text{dim}(\mathrm{\Delta}_{k+1}(\vec{x}_0))
\]
It turns out that for all $\vec{x}_{0} \in \mathcal{E}_{1}$
\[
\mathrm{dim}(<\mathrm{\Delta}|\mathrm{\Delta}_0>) = 8 < n = 9
\]
Hence $R_{T}^{V}(\mathcal{E}_{1})$ is containted in a submanifold of dimension $8$.
\par
Although the system is not s.t.l.a in the points of interest it could be local accessible
in a weaker sense, i.e., without the requirement that the time $T$ is arbitrarily small.
A \emph{necessary} condition for weak local accessibility is that the dimension of the smallest
$\mathrm{\Delta}$-invariant distribution containing $\mathrm{\Delta}$ is equal to $n$.
For the system under examination it can be found that
\[
\mathrm{dim} <\mathrm{\Delta},\mathrm{\Delta}> = 8 < n = 9
\]
for every $\vec{x}_0 \in \mathcal{E}_1$.
\par
In conclusion, at least for the equilibria that belong to $\mathcal{E}_1$, the
system is not locally accessible in any sense hence it can't be locally controllable.

\newpage

\section{Controllability}

\begin{frame}{Small time local controllability \theory}
  Given a point $\vec{x}_0$ and an arbitrarily small neighbourhood $V(\vec{x}_0)$ define the set
  \[
  R_{T}^{V}(\vec{x}_0) = \{\vec{x}(\vec{x_0}, T, \bar{\vec{u}}(t)) \mid
  \vec{x}(\vec{x_0}, \tau, \bar{\vec{u}}(t)) \in V(\vec{x}_0) \enspace \forall \tau \in [0,T]\}
  \]
  with $\bar{\vec{u}}$ appropriate inputs
  \par
  A nonlinear system $\dot{\vec{x}} = \vec{f}(\vec{x}, \vec{u})$ is said \alert{locally-locally controllable} (l.l.c.) from $\vec{x}_0$ if
  \[
  \forall \enspace V(\vec{x}_0) \enspace \exists  \enspace T \enspace
  s.t. \enspace R_{T}^{V}(\vec{x}_0) \enspace \supseteq \enspace B(\vec{x}_0)
  \]
  with $B(\vec{x}_0)$ an open neighbourhood
  \par
  If the time $T$ can be arbitrarily small the system is said \alert{small time locally controllable} (s.t.l.c.)
\end{frame}

\begin{frame}{s.t.l.c and linear controllability \theory}
  \begin{theorem}
    Consider a nonlinear system $\dot{\vec{x}} = \vec{f}(\vec{x}, \vec{u})$ and an equilibrium point $\vec{x}_{eq}$
    such that $\vec{f}(\vec{x}_{eq}, \vec{0}) = \vec{0}$. If the linear approximation of the system around
    the point $\vec{x}_{eq}$ is completely controllable then the system is small-time locally controllable
    in $\vec{x}_{eq}$.
  \end{theorem}
\end{frame}

\begin{frame}{s.t.l.c and linear controllability \cubli}
  \begin{exampleblock}{Controllability of the linear approximation of the Cubli}
  \[
  \begin{split}
    \dot{\vec{\delta x}} &= \frac{\partial{\vec{f}(\vec{x},\vec{u})}}{\partial{\vec{x}}}\Big|_{\mathcal{E}_1} \vec{\delta x} +
    \frac{\partial{\vec{f}(\vec{x},\vec{u})}}{\partial{\vec{u}}}\Big|_{\mathcal{E}_1} \vec{\delta u}\\
    &= A \vec{\delta x} + B \vec{\delta u}
  \end{split}
  \]
  where $\vec{f}(\vec{x},\vec{u})$ does \alert{not} depend on $\theta_1$
  \par
  The pair $(A, B)$ is \alert{not} completely controllable. Indeed
  \[
  rank\left(
  \begin{bmatrix}
    B & AB & A^2B & \dots & A^{n-1}B
  \end{bmatrix}
  \right) = 8 < n = 9
  \]
  \end{exampleblock}
  It is \alert{not possible to conclude} on the small-time local controllability of the system
  in those points belonging to $\mathcal{E}_1$
\end{frame}

\begin{frame}{Small time local accessibility \theory}
  If $R_{T}^{V}(\vec{x}_0)$ only contains an open set of $\vec{x}_0$
  the small time local controllability becomes \alert{small-time local accessibility}
  (s.t.l.a).
  \par
  \[
  \Rightarrow
  \]
  \[
  \text{s.t.l.c} \qquad \text{s.t.l.a}
  \]
  \[
  \centernot \Leftarrow
  \]
\end{frame}

\begin{frame}[shrink=10]{Small time local accessibility (continued) \theory}
  Consider a control affine system
  \[
  \dot{\vec{x}} = \vec{f}(\vec{x}) + \sum\limits_{j=1}^{m}\vec{g}_j(\vec{x}) u_{j}
  \]
  and the distributions
  \[
  \mathrm{\Delta}_0 = \mathrm{span} \{\vec{g}_1, \enspace \hdots \enspace, \vec{g}_m\} \quad
  \quad
  \mathrm{\Delta} = \mathrm{span} \{\vec{f},\enspace\vec{g}_1, \enspace \hdots \enspace, \vec{g}_m\}
  \]
  \begin{theorem}
    Consider the smallest $\mathrm{\Delta}$-invariant distribution containing $\mathrm{\Delta}_0$
    called \alert{accessibility distribution} $<\mathrm{\Delta}|\mathrm{\Delta}_0>$.
    \begin{itemize}
    \item[a.]If the dimension of $<\mathrm{\Delta}|\mathrm{\Delta}_0>= n$ in $\vec{x}_0$ then system
      is \alert{s.t.l.a in $\vec{x}_{0}$};
    \item[b.]If $\mathrm{dim}(<\mathrm{\Delta}|\mathrm{\Delta}_0>)= r < n$ in a neighbourhood of $\vec{x}_0$
      then the set $R_{T}^{V}(\vec{x}_0)$ is contained in a submanifold of dimension $r$ of the
      $n$-dimensional state space, and contains an open set in that submanifold.
    \end{itemize}
  \end{theorem}
\end{frame}

\begin{frame}{Accessibility distribution \theory}
  In order to evaluate $<\mathrm{\Delta}|\mathrm{\Delta}_0>$
  the following filtration of distributions
  \[
  \begin{cases}
    \mathrm{\Delta}_1 &= \mathrm{\Delta}_0 + [\mathrm{\Delta}_0,\mathrm{\Delta}]\\
    &\vdotswithin{=} \\
    \mathrm{\Delta}_k &= \mathrm{\Delta}_{k-1} + [\mathrm{\Delta}_{k-1},\mathrm{\Delta}]
  \end{cases}
  \]
  is performed until an integer $k$ is found s.t.
  \[
  \text{dim}(\mathrm{\Delta}_{k}(\vec{x}_0)) = \text{dim}(\mathrm{\Delta}_{k+1}(\vec{x}_0))
  \]
\end{frame}

\begin{frame}{Accessibility distribution \cubli}
  \begin{exampleblock}{Accessibility distribution of the Cubli}
    Using the Symbolic Math Toolbox from MATLAB it turns out that for all $\vec{x}_{0} \in \mathcal{E}_{1}$ 
    \[
    \begin{split}
      &k = 3\\
      &\mathrm{dim}(<\mathrm{\Delta}|\mathrm{\Delta}_0>) = 8 < n = 9
    \end{split}
    \]
    Hence $R_{T}^{V}(\mathcal{E}_{1})$ is containted in a submanifold of dimension $8$
  \end{exampleblock}
\end{frame}

\begin{frame}{Weak local accessibility \theory}
  Although the system is not s.t.l.a in the points of interest it could be local accessible
  in a weaker sense, i.e., without the requirement that the time $T$ is arbitrarily small.
  \begin{theorem}
    A \alert{necessary} condition for weak local accessibility is
    \[
    \mathrm{dim}(<\mathrm{\Delta}|\mathrm{\Delta}>) = n
    \]
  \end{theorem}
\end{frame}

\begin{frame}{Weak local accessibility \cubli}
  \begin{exampleblock}{Weak local accessibility of the Cubli}
    Using the Symbolic Math Toolbox from MATLAB it turns out that
    \[
    \mathrm{dim} <\mathrm{\Delta} | \mathrm{\Delta}> = 8 < n = 9
    \quad \forall \vec{x}_0 \in \mathcal{E}_1
    \]
  \end{exampleblock}
  \alert{The system is not locally accessible in any sense hence it can't be locally controllable}
\end{frame}

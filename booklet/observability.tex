\section{Observability}
In this section the nonlinear \emph{local} observability of the system is discussed.
The concept of local observability can be explained in terms of the indistinguishability
between a given \emph{initial} state $\bar{\vec{x}}$ and another initial state ``near'' $\bar{\vec{x}}$ namely
$\bar{\vec{x}} + \vec{\delta x}$.
\par
The initial states $\bar{\vec{x}}$ and $\bar{\vec{x}} + \vec{\delta x}$ are said to be
indistinguishable in the interval $[0, T]$ if for every input $\vec{u}$ the outputs
$\vec{y(\bar{\vec{x}}, \vec{u}, t)} = \vec{y(\bar{\vec{x}} + \vec{\delta x}, \vec{u}, t)}$
for every $t \in [0, T]$.
\par
Local indistinguishability can be assessed using appropriate tools from the theory
of nonlinear systems and in some cases using the theory of LTI systems as explained
in the following.

\subsection{Linear Observability}
In order to study the local indistinguishability between two initial states using the theory of LTI systems
the following theorem can be used.
\begin{theorem}
  Consider a nonlinear system 
  \[
  \begin{cases}
    \dot{\vec{x}} = \vec{f}(\vec{x}, \vec{u})\\
    \vec{y} = \vec{h}(\vec{x}, \vec{u})
  \end{cases}
  \]
  and an equilibrium point $\vec{x}_{eq}$ such that $\vec{f}(\vec{x}_{eq}, \vec{0}) = \vec{0}$.
  If the linear approximation of the system around the point $\vec{x}_{eq}$ is completely
  observable then there are no indistinguishable points from $\vec{x}_{eq}$ in a small enough
  neighbourhood of $\vec{x}_{eq}$.
\end{theorem}
It is recalled that a linear system
\[
\dot{\vec{\delta x}} = A \vec{\delta x} + B \vec{\delta u}
\]
\[
\dot{\vec{\delta y}} = C \vec{\delta x}
\]
where
\[
\vec{\delta x}(t) = \vec{x}(t) - \vec{x}_{eq} \quad \vec{\delta u}(t) = \vec{u}(t) - \vec{u}_{eq} \quad
\vec{\delta y}(t) = \vec{y}(t) - \vec{h}(\vec{x}_{eq}, \vec{u}_{eq})
\]
is observable if the rank of the observability matrix $R_o$ 
\begin{equation} \label{eq:linear_observability}
R_o = 
\begin{bmatrix}
C \\ CA \\ CA^2 \\ \vdots \\ CA^{n-1}
\end{bmatrix}
\end{equation}
is equal to the dimension of the state $n$.
\par
For the system of equations (\ref{eq:nl_eq_2}) and (\ref{eq:output}) it can be found that the linear approximation
around each equilibrium point belonging to $\mathcal{E}_{1}$ is not completely observable and
the matrix $R_o$ has rank $6 < n = 9$. As a consequence it is not possible to conclude on
the local indistinguishability relative to those points.

\subsection{Nonlinear Observability}
As done for the nonlinear local controllability the definition of local
indistinguishability should be refined with the requirement that
the inputs $\vec{u}$ are such that the state trajectory never goes outside of
a neighbourhood of the initial states for every $t \in [0, T]$.
The notation used to denote two indistinguishable initial states $\bar{\vec{x}}_1$
and $\bar{\vec{x}}_2$ in the interval $[0,T]$ is
\[
\bar{\vec{x}}_1 I^{U}_{T} \bar{\vec{x}}_2
\]
where $U \subseteq \{\vec{u}(\cdot):[0,T] \rightarrow \mathbb{R}^{m}\}$ contains
input functions which satisfy the hypothesis given above.
\par
In order to study the nonlinear locally indistinguishability of a system written in control affine form
\begin{equation}\label{eq:nonlinear_system_outputs}
  \begin{cases}
    \dot{\vec{x}} = \vec{f}(\vec{x}) + g(\vec{x}) \vec{u}\\
    \vec{y} = \vec{h}(\vec{x}, \vec{u})
  \end{cases}
\end{equation}
the codistribution
\[
\mathrm{\Omega}_0 = \mathrm{span} \{ d \vec{h}\}
\]
and the distribution
\[
\mathrm{\Delta}= \mathrm{span} \{\vec{f},\enspace\vec{g}_1, \enspace \hdots \enspace, \vec{g}_m\}
\]
have to be considered. The following theorem holds
\begin{theorem}\label{th:obs}
  Consider the smallest $\mathrm{\Delta}$-invariant codistribution containing $\mathrm{\Omega}_0$ called
  observability codistribution $< \mathrm{\Delta} | d\vec{h} >$.
  \begin{itemize}
  \item[a.]If the dimension of $< \mathrm{\Delta}|d\vec{h} >$ is equal to $n$ in $\bar{\vec{x}}$ then
    there are no initial states ``near'' $\bar{\vec{x}}$ that are indistinguishable from
    it and the system (\ref{eq:nonlinear_system_outputs}) is said locally observable in $\bar{\vec{x}}$;
  \item[b.]If $\text{dim}(< \Delta|\Omega_0 >) = d < n$ in a neighbourhood of $\bar{\vec{x}}$ then
    the $(n-d)$-dimensional distribution $< \Delta|\Omega_0 >^{\perp}$ that annihilates
    the observability codistribution is involutive. Clearly such distribution evaluated at $\bar{\vec{x}}$
    identifies the displacements $\vec{\delta{x}}$ such that $\bar{\vec{x}} I^{U}_{T}(\bar{\vec{x}} + \vec{\delta x})$
  \end{itemize}
\end{theorem}
\par
In order to evaluate the observability codistribution for the system under examination
(equations \ref{eq:nl_eq_2} and \ref{eq:output}) the following filtration of codistributions
\[
\begin{cases}
\Omega_1 &= \Omega_0 + L_\Delta \Omega_0\\
&\vdotswithin{=} \\
\Omega_k &= \Omega_{k-1} + L_\Delta \Omega_{k-1}
\end{cases}
\]
was performed until an integer $k$ was found such that
\[
\mathrm{dim}(\Delta_{k}(\bar{\vec{x}})) = \mathrm{dim}(\Delta_{k + 1}(\bar{\vec{x}}))
\]
\par
It turns out that for all $\bar{\vec{x}} \in \mathcal{E}_{1}$, i.e.,the cubic frame
is standing still in the upright configuration with zero flywheel velocities,
\[
\mathrm{dim} (<\mathrm{\Delta},d\vec{h}>) = 6 < n = 9
\]
but the codistribution is not regular, i.e., the dimension is not constant in a neighbourhood
of a given $\bar{\vec{x}} \in \mathcal{E}_{1}$ hence the theorem (\ref{th:obs}) cannot be applied.
Conversely any given initial state in which the cubic frame is standing still in the upright
configuration with \emph{non zero angular rates} gives
\[
\mathrm{dim} (<\mathrm{\Delta},d\vec{h}>) = n
\]
hence the system is locally observable in those points.
\newpage

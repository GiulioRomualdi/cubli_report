\section{Observability}

\begin{frame}{Local observability and indistinguishability}
  The concept of local observability can be explained in terms of the indistinguishability
  between two ``near'' \emph{initial} states $\bar{\vec{x}}$ and $\bar{\vec{x}} + \vec{\delta x}$.
  \par
  The initial states $\bar{\vec{x}}$ and $\bar{\vec{x}} + \vec{\delta x}$ are said to be
  indistinguishable in the interval $[0, T]$ if
  \[
  \vec{y}(\bar{\vec{x}}, \vec{u}, t) = \vec{y}(\bar{\vec{x}} + \vec{\delta x}, \vec{u}, t)\enspace
  \forall \vec{u} \in U \enspace \forall t \in [0,T]
  \]
  where $U \subseteq \{\vec{u}(\cdot):[0,T] \rightarrow \mathbb{R}^{m}\}$
  \par
  $\bar{\vec{x}}_1 I^{U}_{T} \bar{\vec{x}}_2$ is the notation used to denote two indistinguishable initial states $\bar{\vec{x}}_1$
  and $\bar{\vec{x}}_2$ in the interval $[0,T]$

\end{frame}

\begin{frame}{Local observability and linear controllability}
  \begin{theorem}
    Consider a nonlinear system 
    \[
    \begin{cases}
      \dot{\vec{x}} = \vec{f}(\vec{x}, \vec{u})\\
      \vec{y} = \vec{h}(\vec{x}, \vec{u})
    \end{cases}
    \]
    and an equilibrium point $\vec{x}_{eq}$ such that $\vec{f}(\vec{x}_{eq}, \vec{0}) = \vec{0}$.
    If the linear approximation of the system around the point $\vec{x}_{eq}$ is completely
    observable then there are no indistinguishable points from $\vec{x}_{eq}$ in a small enough
    neighbourhood of $\vec{x}_{eq}$.
  \end{theorem}
\end{frame}

\begin{frame}{Results}
  For the system under examination the linear approximation
  \[
  \begin{cases}
    \dot{\vec{\delta x}} = \frac{\partial{\vec{f}(\vec{x},\vec{u})}}{\partial{\vec{x}}}\Big|_{\mathcal{E}_1} \vec{\delta x} +
    \frac{\partial{\vec{f}(\vec{x},\vec{u})}}{\partial{\vec{u}}}\Big|_{\mathcal{E}_1} \vec{\delta u} = A \vec{\delta x} + B \vec{\delta u}\\
    \vec{\delta y} = \frac{\partial{\vec{h}(\vec{x})}}{\partial{\vec{x}}}\Big|_{\mathcal{E}_1} \vec{\delta x}= C \vec{\delta x}
  \end{cases}
  \]
  is such that the pair $(A, C)$ is not completely observable. Indeed
  \[
  rank\left(
  \begin{bmatrix}
    C \\ CA \\ CA^2 \\ \vdots \\ CA^{n-1}
  \end{bmatrix}
  \right) = 6 < n = 9
  \]
  It is not possible to conclude on the local observability of the system
  in those points belonging to $\mathcal{E}_1$
\end{frame}

\begin{frame}{Nonlinear local observability}
  \small
  Consider a control affine system, its outputs
  \[
    \dot{\vec{x}} = \vec{f}(\vec{x}) + \sum\limits_{j=1}^{m}\vec{g}_j(\vec{x}) u_{j} \quad \vec{y} = \vec{h}(\vec{x})
  \]
  the codistribution and the distribution
  \[
  \mathrm{\Omega}_0 = \mathrm{span} \{ d \vec{h}\} \quad \mathrm{\Delta}= \mathrm{span} \{\vec{f},\enspace\vec{g}_1, \enspace \hdots \enspace, \vec{g}_m\}
  \]
  \begin{theorem}
    Consider the smallest $\mathrm{\Delta}$-invariant codistribution containing $\mathrm{\Omega}_0$ called
    observability codistribution $< \mathrm{\Delta} | d\vec{h} >$.
    \begin{itemize}
    \item[a.]If the dimension of $< \mathrm{\Delta}|d\vec{h} >$ is equal to $n$ in $\bar{\vec{x}}$ then
      there are no initial states ``near'' $\bar{\vec{x}}$ that are indistinguishable from
      it and the system is said locally observable in $\bar{\vec{x}}$;
    \item[b.]If $\text{dim}(< \Delta|\Omega_0 >) = d < n$ in a neighbourhood of $\bar{\vec{x}}$ then
      the $(n-d)$-dimensional distribution $< \Delta|\Omega_0 >^{\perp}$ that annihilates
      the observability codistribution is involutive.
    \end{itemize}
  \end{theorem}
\end{frame}

\begin{frame}{Results}
In order to evaluate the observability codistribution for the system under examination
 the following filtration of codistributions
\[
\begin{cases}
\Omega_1 &= \Omega_0 + L_\Delta \Omega_0\\
&\vdotswithin{=} \\
\Omega_k &= \Omega_{k-1} + L_\Delta \Omega_{k-1}
\end{cases}
\]
was performed using the Symbolic Math Toolbox from MATLAB until an integer $k$ was found such that
\[
\mathrm{dim}(\Omega_{k}(\bar{\vec{x}})) = \mathrm{dim}(\Omega_{k + 1}(\bar{\vec{x}}))
\]
\end{frame}

\begin{frame}{Results}
It turns out that for all $\bar{\vec{x}} \in \mathcal{E}_{1}$
\[
\mathrm{dim} (<\mathrm{\Delta},d\vec{h}>) = 6 < n = 9
\]
but the codistribution is \alert{not} regular then the theorem cannot be applied.

Conversely any given initial state in which the cubic frame is standing in the upright
configuration with \alert{non zero angular rates} gives
\[
\mathrm{dim} (<\mathrm{\Delta},d\vec{h}>) = n
\]
hence the system \alert{is locally observable} in those points.    
\end{frame}
